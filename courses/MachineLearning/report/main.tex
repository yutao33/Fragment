\documentclass[UTF8]{ctexart}

\usepackage{geometry}
\geometry{a4paper,scale=0.8}

% Title Page
\title{同济大学机器学习课程报告---PCA和ICA理论推导与实际应用}
\author{余海涛 1732985}


\begin{document}
\maketitle

\begin{abstract}
    本篇报告分析了机器学习中使用的主成分分析(PCA)和独立成分分析(ICA)方法的数学原理和具体实现,
    分析了两者之间的相同和不同之处。
\end{abstract}


\section{介绍}
主成分分析(Principal Component Analysis, PCA),\cite{pca}是一种统计方法。通过正交变换将一组可能存在相关性的变量转换为一组线性不相关的变量,转换后的这组变量叫主成分。

\section{PCA理论推导}

\section{ICA理论推导}

\section{实际使用案例}

\section{总结}
本文从数学原理出发分析了PCA和ICA方法的相同和不同之处。通过本文

\bibliography{ref.bib}

\end{document}          
