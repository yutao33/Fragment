本文基于Python平台具体实现了FastICA的算法,核心代码及注释如下:

\begin{minted}{python}
#!/ usr/bin/env python
# FastICA from ICA book , table 8.4

import math
import random
import matplotlib . pyplot as plt
from numpy import *

n_components = 2
def f1(x, period = 4):
    return 0.5*(x- math . floor (x/ period )* period )
def create_data ():
    # data number
    n = 500
    # data time
    T = [0.1*xi for xi in range (0, n)]
    # source
    S = array ([[sin(xi) for xi in T], [f1(xi) for xi in T]],
    float32 )
    #mix matrix
    A = array ([[0.8, 0.2], [-0.3, -0.7]], float32 )
    return T, S, dot(A, S)

def whiten (X):
    # zero mean
    X_mean = X. mean ( axis =-1)
    X -= X_mean [:, newaxis ]
    # whiten
    A = dot(X, X. transpose ())
    D , E = linalg .eig(A)
    D2 = linalg . inv( array ([[D[0], 0.0], [0.0, D[1]]], float32 ))
    D2[0,0] = sqrt (D2[0,0]); D2[1,1] = sqrt (D2[1,1])
    V = dot(D2 , E. transpose ())
    return dot (V, X), V

def _logcosh (x, fun_args =None , alpha = 1):
    gx = tanh ( alpha * x, x); g_x = gx ** 2
    g_x -= 1.; g_x *= - alpha
    return gx , g_x. mean ( axis =-1)

def do_decorrelation (W):
    # black magic
    s, u = linalg . eigh ( dot(W, W.T))
    return dot (dot(u * (1. / sqrt (s)), u.T), W)

def do_fastica (X):
    n, m = X. shape ; p = float (m); g = _logcosh
    # black magic
    X *= sqrt (X. shape [1])
    # create w
    W = ones ((n,n), float32 )
    for i in range (n):
        for j in range (i):
            W[i,j] = random . random ()
    # compute W
    maxIter = 200
    for ii in range ( maxIter ):
        gwtx , g_wtx = g( dot(W, X))
        W1 = do_decorrelation ( dot(gwtx , X.T) / p - g_wtx [:,newaxis ] * W)
        lim = max ( abs (abs ( diag (dot (W1 , W.T))) - 1) )
        W = W1
        if lim < 0. 0001 :
            break
    return W

def show_data (T, S):
    plt. plot (T, [S[0,i] for i in range (S. shape [1])], marker ="*")
    plt. plot (T, [S[1,i] for i in range (S. shape [1])], marker ="o")
    plt. show ()

def main ():
    T, S, D = create_data ()
    Dwhiten , K = whiten (D)
    W = do_fastica ( Dwhiten )
    #Sr: reconstructed source
    Sr = dot( dot(W, K), D)
    show_data (T, D)
    show_data (T, S)
    show_data (T, Sr)

if __name__ == " __main__ ":
    main ()
\end{minted}