\section{理论推导}

本节给出ICA理论的具体数学推导,推导的过程主要参考了\cite{icachinese,ica,fastica,icatutorial}。

\subsection{问题定义}
可以使用统计上的“隐变量”模型定义ICA问题,假设观察到$n$个随机变量$x_1,x_2,\ldots,x_n$,而这些变量是由另外$n$个随机变量$s_1,s_2,\ldots,s_n$线性组合得到,可以用矩阵形式表示如下
\begin{equation}
    \mathbf{x}=\mathbf{As}
\end{equation}
其中
\begin{equation}
    \mathbf{x}=\left(
        \begin{aligned}
        x_1\\
        x_2\\
        \vdots\\
        x_n
        \end{aligned}
    \right),
    \mathbf{s}=\left(
        \begin{aligned}
        s_1\\
        s_2\\
        \vdots\\
        s_n
        \end{aligned}
    \right),
    \mathbf{A}=\left(
        \begin{array}{cccc}
        a_{11} & a_{12} & \cdots & a_{1n} \\ 
        a_{21} & a_{22} & \cdots & a_{2n} \\ 
        \vdots & \vdots & \ddots & \vdots \\ 
        a_{n1} & a_{n2} & \cdots & a_{nn}
        \end{array} 
    \right)
\end{equation}

\textbf{假设和约束}:为了确保ICA模型可以被估计,必须做出下面的假设和约束
\begin{enumerate}
    \item 假设独立成分是统计独立的
    \item 独立成分具有非高斯的分布
    \item 未知的混合矩阵是方阵
\end{enumerate}

\textbf{ICA中的不确定性}:在以上定义的ICA模型中,可以发现存在下面一些不确定性
\begin{enumerate}
    \item 无法确定独立成分的方差
    \item 无法确定独立成分的次序
\end{enumerate}

\textbf{变量的中心化}:不失一般性,我们可以假定所有的混合变量与独立成分均具有零均值。这作为下面推导的一个默认假设,如果实际情况不满足零均值假设,可以进行中心化的预处理:
\[ 
    \mathbf{x}=\mathbf{x'}-E{\mathbf{x'}}
\]
这样独立成分也同时变为零均值的量,因为:
\[
    E{\mathbf{s}}=\mathbf{A}^{-1}E{\mathbf{x}}
\]

\textbf{不相关性和白化}:不相关性是独立性的一个弱化的形式。如果两个随机变量$y_1$和$y_2$的协方差为零,我们说这两个变量是不相关的:
\[
    cov(y_1,y_2)=E\{y_1y_2\}-E\{y_1\}E\{y_2\}=0
\]
如果随机变量是相互独立的,那么他们就是不相关的,但是不相关并不意味着独立。

白化性(whiteness)是指一个零均值的随机向$y$的各个分量具有相同的单位方差且互相不相关,换句话说$y$的协方差矩阵是单位矩阵:
\[
    E\{\textbf{yy}^T\}=\textbf{\mathit{I}}
\]
白化意味着我们将观察得到的数据向量$\mathbf{x}$与某个矩阵$\mathbf{V}$相乘后得到的是一个新的白化的向量:
\[
    \mathbf{z}=\mathbf{Vx}
\]
白化变换总是可实现的,一般可以利用协方差矩阵的特征值分解(EVD):
\[
    E\{\textbf{xx}^T \}=\textbf{EDE}^T
\]
式中,$\mathbf{E}$是$E\{\mathbf{xx}^T \}$的特征向量的正交矩阵,$\textbf{D}$是相应的特征向量的对角矩阵,$D=diag{d_1,d_2,\cdots,d_n}$。这样,白化过程可以利用下面的白化矩阵来实现:
\[
    \mathbf{V}=\mathbf{E}\mathbf{D}^{-1/2}\mathbf{E}^T
\]


\subsection{极大化非高斯性的ICA估计方法}

\textbf{关于非高斯性}:如果两个独立成分$s_1$和$s_2$是高斯分布的,则它们的联合概率密度可以表示为:
\[
p(s_1,s_2)
=\frac{1}{2\pi}exp(-\frac{s_1^2+s_2^2}{2})
=\frac{1}{2\pi}exp(-\frac{{\lVert \mathbf{s} \rVert}^2}{2})
\]


